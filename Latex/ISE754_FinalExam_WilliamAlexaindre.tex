\documentclass[a4paper, 11pt]{article}
\usepackage{comment} % enables the use of multi-line comments (\ifx \fi)
\usepackage{lipsum} %This package just generates Lorem Ipsum filler text.
\usepackage{fullpage} % changes the margin
\usepackage[a4paper, total={7in, 10in}]{geometry}
\usepackage[fleqn]{amsmath}
\usepackage{amssymb,amsthm}  % assumes amsmath package installed
\newtheorem{theorem}{Theorem}
\newtheorem{corollary}{Corollary}
\usepackage{graphicx}
\usepackage{tikz}
\usetikzlibrary{arrows}
\usepackage{verbatim}
\usepackage{listings}
\usepackage{float}
\usepackage{tikz}
    \usetikzlibrary{shapes,arrows}
    \usetikzlibrary{arrows,calc,positioning}

    \tikzset{
        block/.style = {draw, rectangle,
            minimum height=1cm,
            minimum width=1.5cm},
        input/.style = {coordinate,node distance=1cm},
        output/.style = {coordinate,node distance=4cm},
        arrow/.style={draw, -latex,node distance=2cm},
        pinstyle/.style = {pin edge={latex-, black,node distance=2cm}},
        sum/.style = {draw, circle, node distance=1cm},
    }
\usepackage{xcolor}
\usepackage{mdframed}
\usepackage[shortlabels]{enumitem}
\usepackage{indentfirst}
\usepackage{hyperref}
\usepackage{enumitem}
\renewcommand{\thesubsection}{\thesection.\alph{subsection}}

\newenvironment{question}[2][Q]
    { \begin{mdframed}[backgroundcolor=gray!20] \textbf{#1#2} \\}
    {  \end{mdframed}}

% Define solution environment
\newenvironment{solution}
    {\textit{Solution:}}
    {}

\renewcommand{\qed}{\quad\qedsymbol}
%%%%%%%%%%%%%%%%%%%%%%%%%%%%%%%%%%%%%%%%%%%%%%%%%%%%%%%%%%%%%%%%%%%%%%%%%%%%%%%%%%%%%%%%%%%%%%%%%%%%%%%%%%%%%%%%%%%%%%%%%%%%%%%%%%%%%%%%
\begin{document}
%Header-Make sure you update this information!!!!
\noindent
%%%%%%%%%%%%%%%%%%%%%%%%%%%%%%%%%%%%%%%%%%%%%%%%%%%%%%%%%%%%%%%%%%%%%%%%%%%%%%%%%%%%%%%%%%%%%%%%%%%%%%%%%%%%%%%%%%%%%%%%%%%%%%%%%%%%%%%%
\large\textbf{William Alexaindre} \hfill \textbf{Final Exam}   \\
UnityID: wjperezl \hfill ID: 200 672 569 \\
\normalsize Course: ISE 754 - Logistics Engineering \hfill Term: Fall 2025\\
Instructor: Leila Hajibabai \hfill Due Date: $21^{st}$ November, 2025 \\
\noindent\rule{7in}{2.8pt}
%%%%%%%%%%%%%%%%%%%%%%%%%%%%%%%%%%%%%%%%%%%%%%%%%%%%%%%%%%%%%%%%%%%%%%%%%%%%%%%%%%%%%%%%%%%%%%%%%%%%%%%%%%%%%%%%%%%%%%%%%%%%%%%%%%%%%%%%
% question 1
%%%%%%%%%%%%%%%%%%%%%%%%%%%%%%%%%%%%%%%%%%%%%%%%%%%%%%%%%%%%%%%%%%%%%%%%%%%%%%%%%%%%%%%%%%%%%%%%%%%%%%%%%%%%%%%%%%%%%%%%%%%%%%%%%%%%%%%%
\begin{question}{1}

\end{question}

\begin{solution}

\end{solution}
\noindent\rule{7in}{2.8pt}

%%%%%%%%%%%%%%%%%%%%%%%%%%%%%%%%%%%%%%%%%%%%%%%%%%%%%%%%%%%%%%%%%%%%%%%%%
% question 2
%%%%%%%%%%%%%%%%%%%%%%%%%%%%%%%%%%%%%%%%%%%%%%%%%%%%%%%%%%%%%%%%%%%%%%%%%%%%%%%%%%%%%%%%%%%%%%%%%%%%%%%%%%%%%%%%%%%%%%%%%%%%%%%%%%%%%%%%
\begin{question}{2}

\end{question}
\begin{solution}
    
\end{solution}


\noindent\rule{7in}{2.8pt}
%%%%%%%%%%%%%%%%%%%%%%%%%%%%%%%%%%%%%%%%%%%%%%%%%%%%%%%%%%%%%%%%%%%%%%%%%
% question 5
\begin{question}{5}
For our problem we have $L=1500 \text{ Du}$, Travel costs of $3$ Mu/Du and cargo transportation needs per time unit as shown below: 

$$\delta(d)=\begin{cases}
    5,& \text{if } 0\leq d \leq 500\\
    1,& \text{if } 500< d < 1000\\
    5,& \text{if } 1000\leq d \leq 1500
\end{cases}$$
\end{question}

\begin{solution}
    $$CummulativeDemand(d) = CD(d) = \begin{cases}
    5d,& \text{if } 0\leq d \leq 500\\
    2500 + (d-500),& \text{if } 500< d < 1000\\
    3000 + 5(d-1000),& \text{if } 1000\leq d \leq 1500
    \end{cases}$$


    \begin{itemize}[leftmargin=1.3cm]
        \item[A.] The approach that I will use for this part is solve use the integrals in the formulation to get the answer.
        \begin{enumerate}
            \item For this case, the optimal location is given by:
            $$\displaystyle d^* = \frac{\int_{a}^{b} CD(x) \mathrm{d}x + CD(a)\cdot a - CD(b)\cdot b }{CD(a)-CD(b)}$$ which is obtained by setting the areas on both sides of the $d^*$ equal to each other. For our particular problem:

            $$\displaystyle d^* = \frac{\int_{0}^{1500} CD(x) \mathrm{d}x + CD(0)\cdot 0 - CD(1500)\cdot 1500 }{CD(0)-CD(1500)}$$
            $${\displaystyle d^* = \frac{4,125,000+0 - 8,250,000}{0-5500} = 750}$$ 
            $$\boxed{\displaystyle d^* = 750}$$ The best location for the terminal ($n=1$) is at $750$ Du.
            \item 
        \end{enumerate}



    \end{itemize}
\end{solution}



\end{document}